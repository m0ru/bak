
\section{Design as an Artifact}
``Design-science research must produce a viable artifact in the form of a construct, a model, a method, or an instantiation.'' This allows to demonstrate feasibility -- for cases where that wasn't a given yet -- thus making it research (as opposed to routine design).

% This

\section{Problem Relevance}
``The objective of design-science research is to develop technology-based solutions to important and relevant business problems.''

% Relevancy here is with respect to a  ``constituent community'' (e.g. IS practitioners)
% TODO mention Technology Acceptance Model here (and need to define it)? i haven't really done anything based on it, so whatever

\section{Design Evaluation}
``The utility, quality, and efficacy of a design artifact must be rigorously demonstrated via well-executed evaluation methods.''
% This usually requires integration into the usage context (to see if it ``works'' there or is ``good'' in it), the definition of appropriate metrics and gathering of appropriate data. Evaluation provides valueable and necessary feedback for the design iterations (see figure \ref{fig:hevner})

\section{Research Contributions}
``Effective design-science research must provide clear and verifiable contributions in the areas of the design artifact, design foundations, and/or design methodologies.''
% Important here is the novelty of the artifact -- by extending or innovatively (re-)applying previous knowledge -- as well as its generality and significance.

\section{Research Rigor}
``Design-science research relies upon the application of rigorous methods in both the construction and evaluation of the design artifact.''
% This means applying existing foundations and methodologies, using effective metrics and formalising. Note, however, that an overemphasis on rigor can often lead to lower relevance (Lee 1999), as many environments and artifacts defy an excessive formalism (see ``wicked problems'' at footnote \ref{ref:wicked}). %TODO better reference / use glossary entry

\section{Design as a Search Process}
``The search for an effective artifact requires utilizing available means to reach desired ends while satisfying laws in the problem environment.''
% This entails using heuristic search stragegies (e.g. best-practices as starting point) in ght generate/test-cycle (see figure \ref{fig:hevner}) However, again, it might not be possible to formalize or even determine any of these, due to the ``wicked'' (see footnote \ref{ref:wicked}) nature of tackled problems. As a result it might often be necessary to only work on simpler sub-problems, giving up relevancy in turn.

\section{Communication of Research}
``Design-science research must be presented effectively both to technology-oriented as well as management-oriented audiences.''
%For the former the construction and evaluation process are important (e.g. to allow reproduction). For the latter the question boils down to ``Is it worth the effort to use the artifact for my business?''. This can be broken down as ``What knowledge is required?'' respectively ``Who can use it?'', ``How imporant is the problem?'', ``How effective is the solution?'' as well as some details in appendicesto appreciating the work.
