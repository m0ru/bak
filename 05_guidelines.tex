\chapter{<Core>}

\section{Design as an Artifact}
``Design-science research must produce a viable artifact in the form of a construct, a model, a method, or an instantiation.'' This allows to demonstrate feasibility -- for cases where that wasn't a given yet -- thus making it research (as opposed to routine design).

% This

\section{Problem Relevance}
``The objective of design-science research is to develop technology-based solutions to important and relevant business problems.''

% Relevancy here is with respect to a  ``constituent community'' (e.g. IS practitioners)
% TODO mention Technology Acceptance Model here (and need to define it)? i haven't really done anything based on it, so whatever

\section{Design Evaluation}
``The utility, quality, and efficacy of a design artifact must be rigorously demonstrated via well-executed evaluation methods.''
% This usually requires integration into the usage context (to see if it ``works'' there or is ``good'' in it), the definition of appropriate metrics and gathering of appropriate data. Evaluation provides valueable and necessary feedback for the design iterations (see figure \ref{fig:hevner})

\section{Research Contributions}
``Effective design-science research must provide clear and verifiable contributions in the areas of the design artifact, design foundations, and/or design methodologies.''
% Important here is the novelty of the artifact -- by extending or innovatively (re-)applying previous knowledge -- as well as its generality and significance.

\section{Research Rigor}
``Design-science research relies upon the application of rigorous methods in both the construction and evaluation of the design artifact.''
% This means applying existing foundations and methodologies, using effective metrics and formalising. Note, however, that an overemphasis on rigor can often lead to lower relevance (Lee 1999), as many environments and artifacts defy an excessive formalism (see ``wicked problems'' at footnote \ref{ref:wicked}). %TODO better reference / use glossary entry

\section{Research Rigor}
``Design-science research relies upon the application of rigorous methods in both the construction and evaluation of the design artifact.''
% This means applying existing foundations and methodologies, using effective metrics and formalising. Note, however, that an overemphasis on rigor can often lead to lower relevance (Lee 1999), as many environments and artifacts defy an excessive formalism (see ``wicked problems'' at footnote \ref{ref:wicked}). %TODO better reference / use glossary entry

\begin{comment}

## Process

<!--
Flo: "I think it's interesting to describe the actual process, but you should not over-emphasize it. In the end, you came up with a design and an implementation, and that is the artifact you produced.

If you can show multiple iterations of your artifact with 'experiments' evaluating its appropriateness and refinements, fine - but don't zoom into the microscopic level (first I read this, then that, ...).

"

**argument:** feasibility wasn't clear to begin with!!! -> it's research

-->

1. peacemems notifies me of react
2. reading that, i also stumbled across flux
  * flux-article talks about problems with angular/bi-directional data-bindings resonates (same problems when debugging prev prototype) (?)
4. new ulf screens → we’ll need to rewrite (?)
4. rewriting with the old angular setup (angular 2 isn’t production ready)
4. pre-compilation (js, scss) and bundling setup
  * we also switched away from bootstrap, as we'd need to modify it’s styles that heavily anyway
4. actually: when stores and synching via the ws became a thing, started researching flux, ended up stumbling across redux (#342)
  * apparently, we were using flux even before that (see #342). it’s part of a commit from 23 Sep 2015 (1395ba6) and was only used to test with a draft store, so nvm.
  * and compared different implementations ( 	b824aa2)
6. read up on redux and ways to integrate with existing code-base
7. implementation
  * Frankangular - the Migration Process. Reducing angular to a rendering stage.
  * Frankangular - Duplicate imports :{
  * Migration:
      * Reducing bootstrap usage.
      * Promises: $q to native.
   * started with router and core reducers(?)
   * a lot of mocking → smooth collaboration
8. usability tests (?) not really part of architecture

* Meta: möglichst so, dass ich nichts mehr machen soll
* Meta: Write in a way that large parts can be used as WoN-documentation(?)
* Meta: always spell in full first names of female authors for references
* Meta: repeat important points with different wordings.

The Architecture fails somewhat at keeping sync state across tabs, implementing that is a lot of effort on top of it. Theoretically we could serialize and sync the entire state (making sync a lot easier than with angular and flux), but it’s still no Falcor, Relay or Meteor(?) in that regard.

“iteratively identifying deficiencies in prototypes and creatively developing solutions to address them” (Markus et al., 2002)

es6-includes make bundling a lot easier (no endless include lists in index.jsp anymore)

how did we migrate, step-by-step (central redux architecture first, then add components, write import wrapper for won.js, restructure linkeddata-service.js)

medium.js text field

rdfstore-js:

* use it for caching but not as redux store
* accessing it is asynch (reducers are synchronous)
* not all app data is described in rdf

compare with other architectures (angular 1.X, flux, cyclejs’ mvi, elm,...)

how did co-workers deal with it? ease of use?

interaction/integration with project mngmt workflows. e.g. pull-requests, mocking,...

more difficult architectural decisions:

* Routing
* Rdf-store
* Access Management


## Relevant Github-Issues

* [Owner App README](https://github.com/researchstudio-sat/webofneeds/tree/master/webofneeds/won-owner-webapp)
* [The New Code Base Structure - Structure Diagrams, Refactoring and more](https://github.com/researchstudio-sat/webofneeds/issues/151) (#151)
* [Map widget](https://github.com/researchstudio-sat/webofneeds/issues/222)  (#222) and [marker clustering](https://github.com/researchstudio-sat/webofneeds/issues/227) (#227)
  * leafletjs and osm
* [Address forms](https://github.com/researchstudio-sat/webofneeds/issues/226) (#226)
* [Password-retyping unnecessary if reset-via-email works](https://github.com/researchstudio-sat/webofneeds/issues/264) (#264)
* Experiences with contenteditable ([#278](https://github.com/researchstudio-sat/webofneeds/issues/278))
* [Angular 2.0](https://github.com/researchstudio-sat/webofneeds/issues/300) (#300)
* [Precompilation and Tooling (Bundling, CSS, ES6)](https://github.com/researchstudio-sat/webofneeds/issues/314) (#314)
  * bundling, svg sprites, sass, es6,�  - why and how?
  * SASS and BEM. Also address Semantic CSS (!)
* [SVG-sprites](https://github.com/researchstudio-sat/webofneeds/issues/318) (#318)
* [Template Parsing Performance](https://github.com/researchstudio-sat/webofneeds/issues/319) (#319) - jsx
* [Speech-Bubble-CSS](https://github.com/researchstudio-sat/webofneeds/issues/333) (#333)
  * Afair we now use a better version by simply rotating a div with a border.
* [Documentation-generator](https://github.com/researchstudio-sat/webofneeds/issues/337) (#337)
* [Actions/Stores and Synching ](https://github.com/researchstudio-sat/webofneeds/issues/342) (#342)
  * meta: figures in issue need updates
  * this is the issue that triggered the redux research.
  * Redux ~ Elm Architecture ~ CycleJS Model-View-Intent. The parts (Action-Creators / Actions, Reducers, Views). Insights on handling side-effects (e.g. server-side interaction)
  * dealing with rdf-store
* [Routing and Redux](https://github.com/researchstudio-sat/webofneeds/issues/344) (#344)
* [chrome’s security](https://github.com/researchstudio-sat/webofneeds/issues/372) (#372)
* [WebSocket only created before login](https://github.com/researchstudio-sat/webofneeds/issues/381) (#381)
* [Direct link to need](https://github.com/researchstudio-sat/webofneeds/issues/517).
* [nicer urls via html5mode in ui-router](https://github.com/researchstudio-sat/webofneeds/issues/520) (#520)
* [Speed up build](https://github.com/researchstudio-sat/webofneeds/issues/577) (#577) aka "`jspm install` is slow when you need to run it on every build"
* [Page-load performance optimisation](https://github.com/researchstudio-sat/webofneeds/issues/546) (#546)
* [Human-friendly timestamps ](https://github.com/researchstudio-sat/webofneeds/issues/549) (#549) → tick actions
* [Load data selectively](https://github.com/researchstudio-sat/webofneeds/issues/623) (#623) – Paging
* [Flatten content-node of needs](https://github.com/researchstudio-sat/webofneeds/issues/719) (#719)
* [direct link to conversation not working](https://github.com/researchstudio-sat/webofneeds/issues/728) (#728)
* Unify Directives: Overview » Incoming Requests and Matches List
* Unify Directives: Chat and Incoming Request and Outgoing Request
* [Usability Tests of Demonstrator](https://github.com/researchstudio-sat/webofneeds/issues/752) (#752)

\end{comment}


\section{Communication of Research}
``Design-science research must be presented effectively both to technology-oriented as well as management-oriented audiences.''
%For the former the construction and evaluation process are important (e.g. to allow reproduction). For the latter the question boils down to ``Is it worth the effort to use the artifact for my business?''. This can be broken down as ``What knowledge is required?'' respectively ``Who can use it?'', ``How imporant is the problem?'', ``How effective is the solution?'' as well as some details in appendicesto appreciating the work.
