\chapter{Problem Description}

This thesis is part of the over-arching project of crafting an
end-user-friendly
\fnurl{https://www.matchat.org/owner/}{client-application} for the
\fnurl{http://www.webofneeds.org/}{Web of Needs}
(\fnurl{http://sat.researchstudio.at/en/web-of-needs}{related
publications}), short WoN. The main focus was to research ways of
structuring the JavaScript-based client-application; thus it consisted
of researching and experimenting with state-of-the-art web-application
architectures and tooling, adapting and innovating on them for the
particular problem space, as well as identifying a migration path for
updating the existing code-base. To define the requirements, we first
need to take a high-level look over what the Web of Needs is and how
people can interact with it.

% problem-description\\
% * high-level\\
% * for people who aren't web-devs\\
% * pro problem ein satz: "prob ist im browser ld zu verwenden, dazu müssen sie geladen, geparst, gestored werden."\\

% Problemstellung (JS-Basisarchitektur für WoN-Owner App)\\
% as case study in architecture/migration\\

\todo{
define ontologies and rdf\\
\\
node = won-data/document-server\\
}


\section{Web of Needs}\label{web-of-needs}

It is a set of protocols (and reference implementations) that allow
posting documents, for instance describing supply and demand. Starkly
simplified examples would be ``I have a couch to give away'' or ``I'd
like to travel to Paris in a week and need transportation''. These
documents, called ``needs'' can be posted on arbitrary data servers
(called ``WoN-Nodes''). There they're discovered by matching-service,
that continuously crawls the nodes it finds. Additionally, to get faster
results, nodes can notify matchers of new needs. These then get compared
with the ones the matcher already knows about. If it finds a good pair
-- e.g. ``I have a couch to give away'' and ``Looking for furniture for
my living room'' -- the matcher notifies the owners of these needs. They
can then decide whether they want to contact each other. If they send
and accept each other's contact request, they can start chatting with
each other. The protocol in theory can also be used as a base-level for
other interactions, like entering into contracts or transferring money.

% PREVIOUSLY: It is a set of protocols (and reference implementations) that allow posting things like supply and demand (e.g. "I have a couch to give away") online on an arbitrary data server (called WoN-Node). These documents, called "needs", get discovered by a matching-service that notifies the owners of these needs (e.g. when the matcher finds someone that needs the couch offered). The protocols then allow for chatting (or other transactions) between the owners.

\section{Data on WoN-Nodes}\label{data-on-won-nodes}

Needs, connections between them and any events on those connections are
published on the WoN-Nodes in the form of RDF, which stands for
\fnurl{https://en.wikipedia.org/wiki/Resource_Description_Framework}{Resource
Description Framework}. In it, using a variety of different
syntax-alternatives, data is structured as a graph that can be
distributed over multiple (physical) resources. Edges in the graph in
their basic, most primitive form are described by triples of subject
(the start-node), predicates (the edge-type) and object (the
target-node). Note that subject and object need to be Unique Resource
Identifiers (URIs). Additionally, when using URIs, that also are Uniform
Resource Locators (URLs) -- together with the convention to publish data
for an RDF-node at that URL -- data-graphs on multiple servers can
easily be linked with each other, thus making them
\fnurl{https://en.wikipedia.org/wiki/Linked_data}{Linked Data}. This is a
necessary requirement for the Web of Needs, as data is naturally spread
out across several servers, i.e.~WoN-Nodes.

\begin{figure*}
\centering
\begin{verbatim}
<https://node.matchat.org/won/resource/need/7666110576054190000>
<http://purl.org/webofneeds/model#hasBasicNeedType>
<http://purl.org/webofneeds/model#Demand> .

<https://node.matchat.org/won/resource/need/7666110576054190000>
<http://purl.org/webofneeds/model#hasContent>
_:c14n0 .

<https://node.matchat.org/won/resource/need/7666110576054190000>
<http://www.w3.org/1999/02/22-rdf-syntax-ns#type>
<http://purl.org/webofneeds/model#Need> .

_:c14n0
<http://purl.org/dc/elements/1.1/title>
"Transportation Paris-Charles de Gaulle to City Center" .

_:c14n0
<http://purl.org/webofneeds/model#hasTextDescription>
"I’d like to travel to Paris in a week and need \
transportation (e.g. ride-sharing) from the airport \
to the city-center. :)" .
\end{verbatim}
\caption{Excerpt of a need description (N-Triples)}
\label{fig:needtriples}
\end{figure*}

Some example triples taken from a need description could look something
like the ones in figure \ref{fig:needtriples}.

As you can see, this way of specifying triples, called N-Triples, isn't
exactly developer-friendly; the subject is repeated and large parts of
the URIs are duplicate. The short-URIs starting with an underscore (e.g.
\texttt{\_c14n0}) are called blank-nodes and don't have a meaning
outside of a document.

There are several other mark\-up-lan\-gua\-ges res\-pec\-tively se\-ria\-li\-za\-tion-formats
for bett\-er wri\-ting and ser\-ving these tri\-ples, e.g. Tur\-tle, N3, RDF/XML and
JSON-LD. The same ex\-ample, but in Java\-Script Object No\-ta\-tion for Linked Data
(JSON-LD) would read as in figure \ref{fig:needjson}.

\todo{ TODO get syntax-highlighting to work in figures (see comment in .tex) } % \begin{lstlisting}[style=json]}
\begin{figure*}
\centering
\begin{verbatim}
{
  "@id":"need:7666110576054190000",
  "@type":"won:Need",
  "won:hasBasicNeedType":"won:Demand",
  "won:hasContent": {
    "dc:title":
      "Transportation Paris-Charles de Gaulle to City Center",
    "won:hasTextDescription":
      "I’d like to travel to Paris in a week and need transportation \
      (e.g. ride-sharing) from the airport to the city-center . :)"
  },

  "@context":{
     "need": "https://node.matchat.org/won/resource/need/",
     "rdfs":"http://www.w3.org/2000/01/rdf-schema#",
     "dc":"http://purl.org/dc/elements/1.1/",
     "won":"http://purl.org/webofneeds/model#",
     "won:hasBasicNeedType":{
        "@id":"won:hasBasicNeedType",
        "@type":"@id"
     }
  }
}
\end{verbatim}
\caption{Excerpt of a need description (JSON-LD)}
\label{fig:needjson}
\end{figure*}
% \end{lstlisting}

As you can see, JSON-LD allows to nest nodes and to define prefixes (in
the \texttt{@context}). Together this allow to avoid redundancies. The
other serialization-formats are similar in this regard (and are used
between other services in the Web of Needs); however, as JSON-LD also is
valid JSON/JS-code, it was the natural choice for using it for the
JS-based client-application.

\section{WoN-Owner-Application}\label{won-owner-application}

\subsection{Interaction Design}\label{interaction-design}

Among the three services that play roles in the web of needs --
matchers, nodes and owner-applications -- the work I did has its focus
on the latter of these. It provides people a way to interact with the
other services in a similar way that an email-client allows interacting
with email-servers. Through it, people can:

\begin{itemize}
\item
  Create and post new needs. Currently these consist of a simple
  data-structure with a subject, textual description and optional tags
  or location information.
\item
  View needs and all data in them in a human-friendly fashion
\item
  Share links to posts with other people
\item
  Immediately get notified of and see matches, incoming requests and
  chat messages
\item
  Send and accept contact/connection requests
\item
  Write and send chat messages
\end{itemize}

For exploring these interaction, several prototypes -- both paper-based
and (partly) interactive -- had already been designed, the latest of
which was a (graphical) overhaul by Ulf Harr.

\todo{ screens from last prototype }

\subsection{Technical Requirements}\label{technical-requirements}

On the development-side of things, the requirements were:

\todo{"good DX" as requirement. define it}
\begin{itemize}
\item
  Needs to be able to keep data in sync between browser-tabs running the
  JS-client and the Java-based server. This happens through a REST-API
  and websockets. Most messages arrive at the WoN-Owner-Server from the
  WoN-Node and just get forwarded to the client via the websocket. The
  only data directly stored on and fetched from the Owner-Server are the
  account details, which needs belong to an account, its key-pair and
  information on which events have been seen.
\item
  As subject of a research-project, the protocols can change at any
  time. Doing so should only cause minimal refactoring in the
  owner-application.
\item
  In the future different means of interactions between needs --
  i.e.~types need-to-need connections -- will be added. Doing so should
  only cause minimal changes in the application.
\item
  Ultimately the interface for authoring needs should support a wide
  range of ontologies respectively any ontology people might want to use
  for descriptions. Adapting the authoring guys or even just adding a
  few form input widgets should be seamless and only require a few local
  changes.
\item
  We didn't want to deal with the additional hurdles/constraints of
  designing the prototype for mobile-screens at first, but a later
  adaption/port was to be expected. Changing the client application for
  that should require minimal effort.
\end{itemize}

\todo{
TODO why we implemented it js-based:\\
* bandwith\\
* because it’s become somewhat of a wide-spread practice, i.e. “because everybody’s doing so”\\
* because there already was the angular prototype\\
* because it can run on any OS and device\\
status quo: angular app\\
}

The previous iteration of the prototype had already been implemented in
angular-js 1.X. However, the code-base was proving hard to maintain, as
we continuously had to deal with bugs that were hard to track down,
partly because JavaScript's dynamic nature obscured where they lived in
the code and mostly because causality in the angular-app became
increasingly convoluted and hard to understand. The application's
architecture needed an overhaul to deal with these issues, hence this
work you're reading. Thus, additional requirements were:

\begin{itemize}
\item
  Causality in the application is clear and concise to make
  understanding the code and tracking down bugs easier.
\item
  Local changes can't break code elsewhere, i.e.~side-effects are
  minimized.
\item
  Responsibilities of functions and classes are clear and separated, so
  that multiple developers can easily collaborate.
\item
  The current system state is transparent and easily understandable to
  make understanding causality easier.
\item
  Lessens the problems that JavaScript's weakly-typed nature causes,
  e.g.~bugs causing exceptions/errors way later in the program-flow
  instead of at the line where the problem lies.
\end{itemize}

\begin{comment}
    % TODO requirements for a full stack:
in the problem-descripion: list challenges that need to be tackled by web applications:

* seperation of concerns
  * suitability for collaboration
  * reusability of code
* move processing to client / minimal number of requests (justification for js-apps)
* networking
* optimize page load:
  * less http-requests -> bundling
  * smaller size -> minification
  * precompiling templates
* managing dependencies between scripts -> module systems
* simplicity / a low number of concepts / gentle learning curve
* predictability / maintainability


\end{comment}

\todo{
* TODO image: dependency graph in angular application\\
* slide from FB’s flux presentation?\\
* go through old application and do this empirically for a few components and bugs?\\
}




